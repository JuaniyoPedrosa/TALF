\documentclass[11pt]{article}
    \title{\textbf{Práctica Bloque 1}}
\author{Juan Miguel Pedrosa Garrido}
    \date{26 Octubre 2022}
    
    \addtolength{\topmargin}{-3cm}
    \addtolength{\textheight}{3cm}
\usepackage{graphicx}
\begin{document}

\maketitle
\thispagestyle{empty}

\section{Encuentra el conjunto potencia
R³ de R = \{(1,1),(1,2),(2,3),(3,4)\}}

Siendo $R=\{(1,1),(1,2),(2,3),(3,4)\}$, procederemos a obtener R³.
Sinceramente, no me acuerdo muy bien de como se obtiene y en las transparencias no me entero porqué nos apoyamos en $R^-1$. 
Asi que indagando, he encontrado que se puede hacer mediante composición de relaciones.
 Siendo R² = $R\circ R = \{(1,1),(1,3),(2,4),(3,4)\}$
 .A su vez, calculando R³ = $R ^2\circ R = \{(1,1),(1,4),(2,3),(3,4)\}$
\section{Obtén el archivo .TEX que se indica, completa la demostración y responde a la pregunta}
No he podido encontrar el archivo, he usado el comando los siguientes comandos pero ninguno me ha funcionado:
\subparagraph 1. grep $"\\usepackage\{amsthm, amsmath\}" *.tex$
\subparagraph 2. grep $\\usepackage\{amsthm, amsmath\} *.tex$

Pero ninguno de los dos me funcionó

\end{document}

